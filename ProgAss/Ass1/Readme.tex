\documentclass{article}
\usepackage{indentfirst}
\setlength{\parindent}{2em}
\begin{document}
\section*{Programming Assignment 1 - Group 10}
\subsection*{Introduction}
    This program is divided into two parts, the local registry and central treatment department. Each part will have its
    own .exe file. To run this program, one needs to open two terminals. The detailed explanation will be described below 
    the "Hierarchy" part.
\subsection*{Hierarchy}
\subsubsection*{1.Local}
\indent(i) Local Queue\\
\indent(ii) Local IO\\
\indent(iii) Local Hashtable\\
\indent(iv) Local Patient\\
\indent(v) Local Main\\
\indent(vi) Makefile and other input and output files for testing
\subsubsection*{2.CentralHeap}
\indent(i) Central FibHeap\\
\indent(ii) Central IO\\
\indent(iii) Central Hashtable\\
\indent(iv) Central Patient\\
\indent(v) Assignment Queue\\
\indent(vi) Appointment Queue\\
\indent(vii) AList\\
\indent(viii) Central Main\\
\indent(ix) Makefile and other input and output files for testing
\subsection*{How to run this program?}
    \begin{itemize}
        \item As described above, open two terminals, one at local directory and one at central directory.
        \item Run the test.exe in the Local part first, which will give you a Submit.csv file for output. Then, 
              the program will pause, then you need to run the central.exe in Central part.
        \item After the previous step, half a day has been passed. You need to repeat the last step over and over again 
              to simulate the whole process.
        \item When a week has passed, the central part will inform you to produce a weekly report, follow the instruction
              displayed on the terminal.
        \item When a month has passed, the central part will inform you to produce a monthly report, follow the instruction
              displayed on the terminal.
    \end{itemize}
    The whole procedure may be boring, but we want to simulate the time gap between each report from local registry to center.
\end{document}
