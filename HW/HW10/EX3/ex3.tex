\documentclass{article}
\begin{document}
\section*{EX3}
\section*{(i)}
    Assume the \emph{tournament} has $n$ total nodes. Then for any given node, 
it has $n-1$ edges (whether incoming or outcoming) connecting with other $n-1$ nodes.
We have $n$ nodes, so the number of edges in this \emph{tournament} is $n*(n-1)$.
Since there is exactly one edge between every two vertices, each edge is counted
twice. Finally, the number of edges in a \emph{tournament} with $n$ nodes is $n*(n-1)/2$.

\section*{(ii)}
    From the definition of \emph{tournament} and the derivation in (i), if the number 
of edges is fixed, then the number of nodes and the shape of graph are also uniquely
determined. The only difference occurring in various \emph{tournaments} with the same 
number of edges $n$ is the direction of each edge.\\
    Since each edge has two directions, the overall possibility of the \emph{tournament}
is $2^n$.

\section*{(iii)}
    The \emph{tournament} can not be topologically sorted when there are cycles in the graph.\\
    Topological sorting requires a topological order $<=$ on the set of vertices. If $n$
nodes in the \emph{tournament} form a cycle, that is, they are connected heads to tails, 
then we can not find a starting node with incoming order 0. So the topological order $<=$
doesn't exist in that graph. So not all \emph{tournaments} can be topologically sorted.\\    If there is no cycle in the graph, then we can apply DFS or BFS method to find the 
topological order.
\end{document}
